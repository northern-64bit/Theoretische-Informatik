\section{Kolmogorov Komplexität}
\subsection{Theorie}
    \begin{mainbox}{Algorithmen generieren Wörter}
        Sei $\Sigma$ ein Alphabet und $x \in \Sigma^*$. Wir sagen, dass ein Algorithmus $A$ das Wort $x$ \textbf{generiert}, falls $A$ für die Eingabe $\lambda$ die Ausgabe $x$ liefert.
    \end{mainbox}
    Beispiel:
    \begin{lstlisting}[mathescape = true, escapeinside={(*}{*)}]
        $A_n$: 	begin
                for $i$ = 1 to $n$;
                    write(01);
            end
       \end{lstlisting}
       $A_n$ generiert $(01)^n$.

    \begin{mainbox}{Aufzählungsalgorithmus}
        Sei $\Sigma$ ein Alphabet und sei $L \subseteq \Sigma^*$. $A$ ist ein \textbf{Aufzählungsalgorithmus für $L$}, 
        falls $A$  für jede Eingabe $n \in \N \setminus \{0\}$ die Wortfolge $x_1, ...,x_n$ ausgibt, wobei $x_1, ...,x_n$ die kanonisch $n$ ersten Wörter in $L$ sind.
    \end{mainbox}

    \textbf{Aufgabe 2.21}
    
    Beweisen Sie, dass eine Sprache $L$ genau dann rekursiv ist, wenn ein Aufzählungsalgorithmus für $L$ existiert.
    \begin{mainbox}{}
        Das \textbf{Entscheidungsproblem $(\Sigma, L)$} für ein gegebenes Alphabet $\Sigma$ und eine gegebene Sprache $L \subseteq \Sigma^*$ ist, für jedes $x \in \Sigma^*$ zu entscheiden, ob 
        $$x \in L \text{ oder } x \notin L.$$
        Ein Algorithmus $A$ \textbf{löst} das Entscheidungsproblem $(\Sigma, L)$, falls für alle $x \in \Sigma^*$ gilt:
        $$A(x) = \begin{cases}
            1, &\text{falls }x \in L,\\
            0, &\text{falls }x \notin L.
        \end{cases}$$
        Wir sagen auch, dass $A$ die Sprache $L$ erkennt.
    \end{mainbox}

    $L$ rekursiv $(\implies)$ existiert Aufzählungsalgorithmus:

    Sei $A$ ein Algorithmus, der $L$ erkennt. Wir beschreiben nun einen Aufzählungsalgorithmus $B$ konstruktiv. 

    \begin{algorithm}[H]
        \caption{$B(\Sigma, n)$}
        \begin{algorithmic}
            \State i $\leftarrow$ 0
            \While{i $\leq n$}
                \State $w$ $\leftarrow$ kanonisch nächstes Wort über $\Sigma^*$
                \If{$A(w) = 1$}
                \State print($w$)
                \State $i \leftarrow i + 1$
                \EndIf 
            \EndWhile
        \end{algorithmic}
    \end{algorithm}

    Aufzählungsalgorithmus $B$ $\implies$ $L$ rekursiv:

    \begin{algorithm}[H]
        \caption{$A(\Sigma, w)$}
        \begin{algorithmic}
            \State$n \leftarrow$ $|\Sigma|^{|w| + 1}$
            \State $L \leftarrow B(\Sigma, n)$
            \If{$w \in L$}
            \State print(1)
            \Else
            \State print(0)
            \EndIf
        \end{algorithmic}
    \end{algorithm}
    Es gibt ein kleines Problem. $B$ könnte unendlich lange laufen, falls $n > |L|$.\\ 
    Es sollte nicht so schwierig sein, $B$ zu modifizieren, dass es die Berechnung aufhört, falls es keine weiteren Wörter in $L$ gibt.



    \myTitle{Information messen}

    Wir beschränken uns auf $\Sigma_{\text{bool}}$
    \begin{mainbox}{Kolmogorov-Komplexität}
        Für jedes Wort $x \in (\Sigma_{\text{bool}})^*$ ist die \textbf{Kolmogorov-Komplexität $K(x)$ des Wortes $x$} das Minimum der binären Längen, der Pascal-Programme, die $x$ \comm{generieren}.
    \end{mainbox}
    $\implies$ \comm{Kolmogorov Komplexität eines Wortes ist die Länge des kürzesten Programms, dass keinen Input nimmt und das Wort ausgibt!}

    $K(x)$ ist die kürzestmögliche Länge einer Beschreibung von $x$.

    Die einfachste (und triviale) Beschreibung von $x$, ist wenn man $x$ direkt angibt.

    $x$ kann aber eine Struktur oder Regelmässigkeit haben, die eine Komprimierung erlaubt.

    Welche Programmiersprache gewählt wird verändert die Kolmogorov-Komplexität nur um eine Konstante. (Satz 2.1)



    \textbf{Beispiel}
    
    Sei $w = 0101010101010101010101010101010101010101$. Die Länge von $w$ ist $|w| = 40$ und die triviale Beschreibunglänge wäre wie gegeben $40$.

    Aber durch die Regelmässigkeit von einer $20$-fachen Wiederholung der Sequenz $01$, können $w$ auch durch $(01)^{20}$ beschreiben. 
    Hierbei ist die Beschreibungslänge ein wenig mehr als $4$ Zeichen.



    \textbf{Grundlegende Resultate}
    \begin{mainbox}{}
        Es existiert eine Konstante $d$, so dass für jedes $x \in (\Sigma_{\text{bool}})^*$
        $$K(x) \leq |x| + d$$
    \end{mainbox}

    \begin{mainbox}{Definition!}
        Die \textbf{Kolmogorov-Komplexität einer natürlichen Zahl $n$} ist $K(n) = K(\text{Bin}(n))$.
    \end{mainbox}

    
    \begin{mainbox}{Lemma 2.5 - Nichtkomprimierbar}
        Für jede Zahl $n \in \N \setminus\{0\}$ existiert ein Wort $w_n \in (\Sigma_{\text{bool}})^n$, so dass 
        $$K(w_n) \geq |w_n| = n$$
    \end{mainbox}
    \textbf{Beweis}

    Es gibt $2^n$ Wörter $x_1, ..., x_{2^n}$ über $\Sigma_{bool}$ der Länge $n$. 
    Wir bezeichnen $C(x_i)$ als den Bitstring des kürzesten Programms, der $x_i$ generieren kann. Es ist klar, dass für $i \neq j: C(x_i) \neq C(x_j)$.

    Die Anzahl der nichtleeren Bitstrings, i.e. der Wörter der Länge $< n$ über $\Sigma_{bool}$ ist:
    $$\sum_{i = 1}^{n-1} 2^i = 2^n - 2 < 2^n$$
    Also muss es unter den Wörtern $x_1, ...,x_{2^n}$ mindestens ein Wort $x_k$ mit $K(x_k) \geq n$ geben.
    
    \hspace*{0pt}\hfill$\blacksquare$



    \textbf{Satz 2.1 - Programmiersprachen}

    Für jedes Wort $x \in (\Sigma_{\text{bool}})^*$ und jede Programmiersprache $A$ sei $K_A(x)$ die Kolmogorov-Komplexität von $x$ bezüglich der Programmiersprache $A$.
    \begin{mainbox}{}
        Seien $A$ und $B$ Programmiersprachen. Es existiert eine Konstante $c_{A,B}$, die nur von $A$ und $B$ abhängt, so dass 
        $$|K_A(x) -K_B(x)|\leq c_{A,B}$$
        für alle $x \in (\Sigma_{\text{bool}})^*$.
    \end{mainbox}
    \textit{Beweis im Buch/Vorlesung}

    \begin{mainbox}{Ein zufälliges Wort}
        Ein Wort $x \in (\Sigma_{\text{bool}})^*$ heisst \textbf{zufällig}, falls $K(x) \geq |x|$. 
        
        Eine Zahl $n$ heisst \textbf{zufällig}, falls $K(n) = K(\text{Bin}(n)) \geq \left\lceil\log_2(n + 1) \right\rceil - 1$.
    \end{mainbox}
    Jede Binär-Darstellung beginnt immer mit einer $1$, deshalb können wir die Länge der Binär-Darstellung um $1$ verkürzen.

    Zufälligkeit hier bedeutet, dass ein Wort völlig unstrukturiert ist und sich nicht komprimieren lässt. Es hat nichts mit Wahrscheinlichkeit zu tun.

    \begin{mainbox}{Satz 2.2}
        Sei $L$ eine Sprache über $\Sigma_{\text{bool}}$. Sei für jedes $n \in \N \setminus \{0\}$, $z_n$ das $n$-te Wort in $L$ bezüglich der kanonischen Ordnung. 
        Wenn ein Programm $A_L$ existiert, das das Entscheidungsproblem $(\Sigma_{\text{bool}}, L)$ löst, dann gilt für alle $n \in \N\setminus\{0\}$, dass
        $$K(z_n) \leq \left\lceil\log_2(n+1)\right\rceil + c$$
        wobei $c$ eine von $n$ unabhängige Konstante ist.
    \end{mainbox}
    \textbf{Beweisidee}

    Wir können aus $A_L$, ein Programm entwerfen, dass das kanonisch $n$-te Wort generiert, indem wir in der kanonischen Reihenfolge alle Wörter $x \in (\Sigma_{bool})^*$ durchgehen und mit $A_L$ entscheiden, ob $x \in L$. Dann können wir einen Counter $c$ haben und den Prozess abbrechen, wenn der Counter $c = n$ wird und dann dieses Wort ausgeben.

    Wir sehen, dass dieses Programm ausser der Eingabe $n$ immer gleich ist. Sei die Länge dieses Programms $c$, dann können wir für das $n$-te Wort der Sprache $L, z_n,$ die Kolmogorov-Komplexität auf $n$ reduzieren, bzw:
    $$K(z_n) \leq \lceil \log_2(n+1)\rceil + c$$
    \hspace*{0pt}\hfill$\blacksquare$



    \textbf{Primzahlsatz}

    Für jede positive ganz Zahl $n$ sei Prim($n$) die Anzahl der Primzahlen kleiner gleich $n$. 
    \begin{mainbox}{}
        $$\lim_{n \to \infty}\frac{\text{Prim}(n)}{n/\ln n} = 1$$
    \end{mainbox}
    Nützliche Ungleichung
    $$\ln n - \frac{3}{2} < \frac{n}{\text{Prim(n)}} < \ln n - \frac{1}{2}$$
    für alle $n \geq 67$.

    \begin{mainbox}{Lemma 2.6 - schwache Version des Primzahlsatzes}
        Sei $n_1, n_2, n_3, ...$ eine steigende unendliche Folge natürlicher Zahlen mit $K(n_i) \geq \lceil \log_2 n_i \rceil / 2$. Für jedes $i \in \N \setminus \{0\}$ sei $q_i$ die grösste Primzahl, die die Zahl $n_i$ teilt. Dann ist die Menge $Q = \{q_i \mid i \in \N \setminus\{0\}\}$ unendlich.
    \end{mainbox}
    \textbf{Beweis: }
    Wir beweisen diese Aussage per Widerspruch:

    Nehmen wir zum Widerspruch an, dass die Menge $Q = \{q_i \mid i \in \N \setminus \{0\}\}$ sei endlich. 
    
    Sei $q_m$ die grösste Primzahl in $Q$. Dann können wir jede Zahl $n_i$ eindeutig als 
    $$n_i = p_1^{r_{i, 1}} \cdot p_2^{r_{i,2}} \cdot \cdots \cdot p_m^{r_{i,m}}$$
    für irgendwelche $r_{i,1}, r_{i,2}, ..., r_{i,m} \in \N$ darstellen. 
    
    \comm{Bemerke das die \(p_i\) ausser \(p_m\) nicht notwendigerweise in \(Q\) sein müssen, wir verwenden nur den Fakt, dass es endlich viele davon hat.}
    
    Sei $c$ die binäre Länge eines Programms, dass diese $r_{i,j}$ als Eingaben nimmt und $n_i$ erzeugt (A ist für alle $i\in \N$ bis auf die Eingaben $r_{i,1}, ..., r_{i,m}$ gleich).
    
    Dann gilt:
    $$K(n_i) \leq c + 8 \cdot (\lceil \log_2(r_{i,1}+1)\rceil + \lceil \log_2(r_{i,2}+1)\rceil + ... + \lceil \log_2(r_{i,m}+1)\rceil)$$
    Die multiplikative Konstante $8$ kommt daher, dass wir für die Zahlen $r_{i,1}, r_{i,2}, ..., r_{i,m}$ dieselbe Kodierung, wie für den Rest des Programmes verwenden (z.B. ASCII-Kodierung), damit ihre Darstellungen eindeutig voneinander getrennt werden können. Weil  $r_{i,j} \leq \log_2 n_i, \forall j \in \{1, ..., m\}$ erhalten wir 
    $$K(n_i) \leq c + 8m \cdot \lceil \log_2(\log_2 n_i + 1)\rceil, \forall i \in \N \setminus \{0\}$$
   
    Weil $m$ und $c$ Konstanten unabhängig von $i$ sind, kann 
    $$\lceil \log_2 n_i \rceil / 2 \leq K(n_i) \leq c + 8m \cdot \lceil \log_2(\log_2 n_i + 1)\rceil$$
    $$\lceil \log_2 n_i \rceil/2 \leq c + 8m \cdot \lceil \log_2(\log_2 n_i + 1)\rceil$$
    nur für endlich viele $i \in \N \setminus \{0\}$ gelten. 

    Dies ist ein Widerspruch! 

    Folglich ist die Menge $Q$ unendlich.

    \hspace*{0pt}\hfill$\blacksquare$



\subsection{How To Kolmogorov}



    \begin{subbox}{Aufgabentyp 1}

    Sei $w_n = (010)^{3^{2n^3}} \in \{0,1\}^*$ für alle $n \in \N \setminus \{0\}$. Gib eine möglichst gute obere Schranke für die Kolmogorov-Komplexität von $w_n$ an, gemessen in der Länge von $w_n$.
    \end{subbox}

    \textbf{Lösung Typ 1}
    
    Wir zeigen ein Programm, dass $n$ als Eingabe nimmt und $w_n$ druckt:
    \begin{lstlisting}[language= Pascal, mathescape = true, escapeinside={(*}{*)}]
        $W_n$:     	begin
                $M := n$;
                $M := 2 \times M \times M \times M$;
                $J := 1$;
                for $I$ = 1 to $M$                  
                    $J$ := $J \ \times \ $3;
                for $I$ = 1 to $J$;
                    write(010);
            end
    \end{lstlisting}

    Der einzige variable Teil dieses Algorithmus ist $n$. Der restliche Code ist von konstanter Länge. Die binäre Länge dieses Programms kann von oben durch 
    $$\left\lceil\log_2(n+1)\right\rceil + c$$
    beschränkt werden, für eine Konstante $c$.

    Somit folgt
    $$K(w_n) \leq \log_2(n) + c'$$
    Wir berechnen die Länge von $w_n$ als $|w_n| = |010| \cdot 3^{2n^3} = 3^{2n^3+1}$. 
    
    Mit ein wenig umrechnen erhalten wir $$n = \sqrt[3]{\frac{\log_3|w_n| - 1}{2}}$$
    und die obere Schranke
    $$K(w_n) \leq \log_2 \left(\sqrt[3]{\frac{\log_3|w_n| - 1}{2}}\right)+ c' \leq \log_2 \log_3 |w_n| + c''$$

\vspace*{1cm}

    \begin{subbox}{Aufgabentyp 2}

    Geben Sie eine unendliche Folge von Wörtern $y_1 < y_2 < ...$ an, so dass eine Konstante $c \in \N$ existiert, so dass für alle $i \geq 1$ 
    $$K(y_i) \leq \log_2\log_2\log_3\log_2(|y_i|) + c$$
    \end{subbox}

    \textbf{Lösung Typ 2}
    
    Wir definieren die Folge $y_1,y_2,...$ mit $y_i = 0^{2^{3^{2^i}}}$ für alle $i \in \N$. Da $|y_i| < |y_{i+1}|$ folgt die geforderte Ordnung.

    Es gilt
    $$i = \log_2\log_3\log_2|y_i|  \text{ für } i \geq 1$$

    Wir zeigen ein Programm, dass $i$ als Eingabe nimmt und $y_i$ druckt:

    \begin{lstlisting}[language= Pascal, mathescape = true, escapeinside={(*}{*)}]
        begin
            $M := i$;
            $M := $2^(3^(2^$M$));
            for $I$ = 1 to $M$;
                write(010);
        end
    \end{lstlisting}
    Das \^ \ für die Exponentiation ist nicht Teil der originalen Pascal Syntax, aber wir verwenden es um unser Programm lesbarer zu machen.

    Der einzige variable Teil dieses Programms ist das $i$. Der Rest hat konstante Länge. Demnach kann die Länge diese Programms für eine Konstante $c'$ durch 
    $$\left\lceil\log_2(i+1)\right\rceil + c'$$
    von oben beschränkt werden.

    Somit folgt
    \begin{align*}
        K(y_i) &\leq \log_2(i) + c\\
        &\leq \log_2\log_2\log_3\log_2|y_i| + c
    \end{align*}
    für eine Konstante $c$. 

\vspace*{1cm}

    \begin{subbox}{Aufgabentyp 3}

    Sei $M = \{7^i \mid i \in \N, \ i \leq 2^n-1\}$. Beweisen Sie, dass mindestens sieben Achtel der Zahlen in $M$ Kolmogorov-Komplexität von mindestens $n-3$ haben.

    \end{subbox}
    \textbf{Lösung Typ 3}

    Wir zeigen, dass höchstens $\frac{1}{8}$ der Zahlen $x \in M$ eine Kolmogorov-Komplexität $K(x) \leq n-4$ haben. 

    Nehmen wir zum Widerspruch an, dass $M$ mehr als $\frac{1}{8}|M|$ Zahlen $x$ enthält, mit $K(x) \leq n-4$. 

    Die Programme, die diese Wörter generieren, müssen paarweise verschieden sein, da die Wörter paarweise verschieden sind.

    Es gibt aber höchstens 
    $$\sum_{k=0}^{n-4}2^k = 2^{n-3} - 1 < \frac{1}{8}|M|$$
    Bitstrings mit Länge $\leq n-4$. \textbf{Widerspruch.}



    \vspace*{1cm}

